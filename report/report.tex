%! TEX program =      xelatex
%! TEX bibliography = biber
%%%%%%%%%%%%%%%%%%%%%%%%%%%%%%%%%%%%%%%%%
% Journal Article
% LaTeX Template
% Version 1.4 (15/5/16)
%
% This template has been downloaded from:
% http://www.LaTeXTemplates.com
%
% Original author:
% Frits Wenneker (http://www.howtotex.com) with extensive modifications by
% Vel (vel@LaTeXTemplates.com)
%
% License:
% CC BY-NC-SA 3.0 (http://creativecommons.org/licenses/by-nc-sa/3.0/)
%
%%%%%%%%%%%%%%%%%%%%%%%%%%%%%%%%%%%%%%%%%

%----------------------------------------------------------------------------------------
%	PACKAGES AND OTHER DOCUMENT CONFIGURATIONS
%----------------------------------------------------------------------------------------

% \documentclass[twocolumn]{article}
\documentclass{article}
% allows for temporary adjustment of side margins
\usepackage{chngpage}
\usepackage{multirow}

\usepackage{blindtext} % Package to generate dummy text throughout this template

\usepackage[T1]{fontenc} % Use 8-bit encoding that has 256 glyphs
% \linespread{1.15} % Line spacing - Palatino needs more space between lines
\usepackage{microtype} % Slightly tweak font spacing for aesthetics

\usepackage[english]{babel} % Language hyphenation and typographical rules

% \usepackage[hmarginratio=1:1,top=32mm,columnsep=20pt]{geometry} % Document margins
% \usepackage[hang, small,labelfont=bf,up,textfont=it,up]{caption} % Custom captions under/above floats in tables or figures
\usepackage{booktabs} % Horizontal rules in tables

\usepackage{lettrine} % The lettrine is the first enlarged letter at the beginning of the text

\usepackage{enumitem} % Customized lists
\setlist[itemize]{noitemsep} % Make itemize lists more compact

\usepackage{abstract} % Allows abstract customization
\renewcommand{\abstractnamefont}{\normalfont\huge} % Set the "Abstract" text to bold
\renewcommand{\abstracttextfont}{\normalfont\small\itshape} % Set the abstract itself to small italic text

\usepackage{titlesec} % Allows customization of titles
% \renewcommand\thesubsection{\Roman{subsection}} % roman numerals for subsections
\titleformat{\section}[block]{\Large\scshape\centering}{\thesection}{1em}{} % Change the look of the section titles
% \titleformat{\subsection}[block]{\large}{\thesubsection}{1em}{} % Change the look of the section titles
% \titleformat{\subsubsection}[block]{\large}{\thesubsubsection}{1em}{} % Change the look of the section titles

\usepackage{fancyhdr} % Headers and footers
\fancyhead{} % Blank out the default header
\fancyfoot{} % Blank out the default footer
\fancyfoot[RO,LE]{\thepage} % Custom footer text

\usepackage{titling} % Customizing the title section

\usepackage{hyperref} % For hyperlinks in the PDF
\usepackage[bibstyle=numeric,citestyle=authoryear,backend=biber,natbib=true,maxcitenames=2]{biblatex}
\usepackage{amssymb}



\newrobustcmd*{\parentexttrack}[1]{%
	\begingroup
	\blx@blxinit
	\blx@setsfcodes
	\blx@bibopenparen#1\blx@bibcloseparen
\endgroup}

\AtEveryCite{%
	\let\parentext=\parentexttrack%
	\let\bibopenparen=\bibopenbracket%
\let\bibcloseparen=\bibclosebracket}
\addbibresource{bibliography.bib}

%----------------------------------------------------------------------------------------
%	TITLE SECTION
%----------------------------------------------------------------------------------------

\setlength{\droptitle}{-4\baselineskip} % Move the title up

\title{Face recognition through RGB-D images and matrices of SVMs} % Article title
\author{%
	\textsc{Federico Simonetta}\thanks{Ingegneria Informatica LM, matricola 1129912} \\[1ex] % Your name
	\normalsize \href{mailto:simonettaf@dei.unipd.it}{simonettaf@dei.unipd.it} % Your email address
	\and % Uncomment if 2 authors are required, duplicate these 4 lines if more
	\textsc{Alberto Cenzato}\thanks{Ingegneria Informatica LM, matricola 1134707} \\[1ex] % Your name
	\normalsize \href{mailto:alberto.cenzato@studenti.unipd.it}{alberto.cenzato@studenti.unipd.it} % Your email address
}
\date{\today} % Leave empty to omit a date
\renewcommand{\maketitlehookd}{%
	\begin{abstract}
		\noindent We tried to reimplement the algorithm described in \citep{Hayat2016}. We found that their results are reproducible with large datasets while with small datasets the SVM models have very low precision. We optimized the algorithm for two datasets of different size and tried several different preprocessing tweaks and decision algorithms. Lastly, we state that the biggest contributions to the good training of the SVMs are given by the size and the noiseness of the dataset and the accuracy of the preprocessing steps.
	\end{abstract}
}


\begin{document}

% Print the title
\Huge
\bfseries
\maketitle
\mdseries
\normalsize

\section{Introduction}\label{sec:intro}
\lettrine[nindent=0em,lines=2]{I}n \citep{Hayat2016} an algorithm for face recognition based on RGB-D image is given. RGB-D images are a state-of-the-art scene representation paradigm through which a 3D scene is described. Compared to classical RGB images, the RGB-D paradigm adds the depth information that gives the chance to study images in a three-dimensional space. In their paper, Hayat et al.\ show that the depth information can really raise precision of face recognition algorithms. \\
We tried to reproduce their results using a small dataset consisting of 338 images of 26 different individuals. Because of the very poor results, we tried to train the model with a new dataset of 20 different peoples in 24 sequences for a total of 15.678 RGB-D images \citep{Fanelli2013}. In \citep{Hayat2016}, authors used the second of these datasets, merged with other two datasets for a total of more than 35.000 images. \\ 
Our results let us think that this algorithm needs a very huge dataset for training.

\section{Algorithm description}\label{sec:algorithm_desc}
The algorithm proposed by \citep{Hayat2016} is roughly divided in three steps:
\begin{itemize}
	\item Detect the face and estimate its pose;
	\item Find a suitable feature space for the faces;
	\item Train the SVMs with the feature vectors obtained in the previous step.
\end{itemize}
These steps are briefly described in the next paragraphs. For a complete description see \citep{Hayat2016}.

	\subsection{Normalization}\label{sec:normalization}
	The input RGB-D images are preprocessed with a three steps pipeline. First step in preprocessing is the background removal through a simple \textit{k}-means clustering procedure on depth images, the nearest cluster is the person. Then a face detection and pose estimation algorithm is used \citep{Fanelli2013}. Finally the pose information is used to precisely crop the face. \\
	The basic idea is to look for the first non empty row from top in the segmented depth image, where \textit{non empty} means with at least $m$ non-zero pixels; this is the top of the face ROI, lets call it $y_{top}$. $y_{top}$ is then adjusted adding a factor proportional to the estimated head pose $$y_t = y_t + (\beta \phi +\gamma \psi)$$ where $\phi$ and $\psi$ are the \textit{roll} and \textit{yaw} angles returned by the pose estimation of which in \citep{Fanelli2011}, while $\beta$ and $\gamma$ are parameters set to $5/8$. The ROI height is proportionally inverse to the distance of the head from the camera $$y_{bottom} = y_{top} + 100/z$$
	The cropping window position along the \verb|X| axis is computed by looking for the first non empty column starting from right and the first non empty column from left. \\

	\subsection{Image subsets}
	Once the cropped faces are available they are clustered based on their pose. For each person $c$ pose cluster centers are computed by \textit{k}-means; using these cluster centers, images of the person are assigned to one of the $c$ clusters based on the shortest euclidean distance between the cluster centers and the pose vector of each image.

	\subsection{Image subsets representation}
	Given a cluster of images from the same person each image in the subset is divided into $4 \times 4$ non-overlapping blocks. The image subset is represented by a $\mathbb{R}^{16 \times 16}$ covariance matrix of these blocks\footnote{See \citep{Hayat2016} for further details on how the covariance matrix is computed.}.

	\subsection{SVMs}
	The covariance matrices are then used as a feature vector for the SVMs. $c \times N$ SVMs are trained (where $N$ is the number of people in the dataset) for the covariance matrices of the RGB images and just as many for the covariance matrices of the depth images. All SVMs are binary classification SVMs trained in a one-vs-all fashion. \\
	The SVMs use the Stein kernel function to map the covariance matrices (symmetric positive definite matrices) from Riemannian manifold to a high dimensional Hilbert space where they are linearly separable. The Stein kernel function is defined as $$k(X,Y) = e^{-\sigma (\log{\det{\frac{X+Y}{2}}} - \frac{1}{2}\log{\det{XY}})}$$

	\subsection{Fusion of results}
	Once the SVMs are trained a query image set can be submitted to the algorithm. The image set must be of RGB and depth image all of the same person. These images are preprocessed as in the training. The resulting $2c$ covariance matrices ($c$ for RGB and $c$ for depth images) are classified by the $2c$ SVMs of each training person. The identity which receives the maximum number of "votes" is the predicted identity.


\section{Algorithm implementation}\label{sec:algorithm_impl}
Our goal was to reproduce the results obtained by \citep{Hayat2016}. We implemented the algorithm in \verb|C++| and used the source code for random forest head pose estimation by \citep{Fanelli2013} as done by the authors. \verb|OpenCV| 3 was used both for image processing and SVM training and prediction. Two different implementations are the result of this work, one for each of the datasets we were provided with; they are more or less the same software with major differences only in the preprocessing steps which were tuned for the specific dataset they were made for.

	\subsection{Datasets}\label{sec:datasets}
	The first attempt was to adapt the algorithm by \citep{Hayat2016} on a small dataset containing 338 images of 26 different persons. This dataset contained:
	\begin{itemize}
		\item frontal faces at 1 meter from the sensor;
		\item frontal faces at 2 meters from the sensor;
		\item non frontal faces at 1 meter from the sensor;
		\item frontal faces at 1 meter from the sensor with different expressions;
		\item variations in lighting;
		\item changing background scenes.
	\end{itemize}
	The very poor results obtained on this dataset forced us to implement all tweaks described in \ref{sec:background}, \ref{sec:cropping}, \ref{sec:training}. Anyway, our results did not agree with the results obtained in the paper. We can state with a reasonable certainty that the dataset was too small to train the model. \\
	Therefore, we tested the algorithm on a new dataset by \citep{Fanelli2013}, containing almost half of the images used in \citep{Hayat2016}, with more than 15.000 images of 20 different persons. On this new dataset, depth images were already segmented, therefore we removed the background segmentation step from the preprocessing pipeline. \\
	From now on we will refer to the first dataset as \verb|dataset A| and the second as \verb|dataset B|.

	\subsection{Background removal} \label{sec:background}
	The first step in the preprocessing pipeline is to remove the background: a clean depth image, with the person only, should be fed into the face cropping step (see \ref{sec:cropping}). Most of the effort in the implementation of the algorithm for \verb|dataset A| was put in this step. The task was not easy because of the non-uniform background in the images: the background was made of many objects and people at different depths; it was not feasible to simply use a \textit{k}-means clustering as in \cite{Hayat2016} because the image did not have two obvious clusters. \\
	In\verb|dataset B| this was already done on depth images, so we have been able to skip this step. \\

		\subsubsection{Face detection first}
		A common error of \textit{k}-means clustering on \verb|dataset A| was to consider some objects and people slightly behind the individual of interest as part of his/her cluster. This produced a too loose cropping threshold resulting in an image with a partially removed background and making the pose estimation step \ref{sec:cropping} fail. We tried to overcome this introducing a face-detection step before clustering: only the face ROI was used for \textit{k}-means reducing the probability of clustering the wrong objects. If no face was detected, a fixed threshold segmentation was performed. \\
		We used the Haar Cascades classifiers provided by OpenCV\footnote{For a presentation of the method see \href{https://docs.opencv.org/3.3.0/d7/d8b/tutorial_py_face_detection.html}{https://docs.opencv.org/3.3.0/d7/d8b/ \\
		/tutorial\_py\_face\_detection.html} and \citep{ViolaJones}} to detect the face. \\
		This method has three major disadvantages:
		\begin{itemize}
			\item As known Haar Cascades classifiers are good at detecting frontal faces but their performance quickly decreases when processing non frontal or partially occluded faces. Moreover when many people are in the image it is difficult to say which one is the individual of interest;
			\item Of course it adds a computational cost;
			\item It is inefficient and redundant to perform a face detection in this phase since it will be done anyway in the pose estimation step \ref{sec:cropping}.
		\end{itemize}

		\subsubsection{Outlier removal}
		After \textit{k}-means was applied all the background was removed, but many outliers to the right and the left of the person remained.
		We traversed the image as a graph to find the connected components using OpenCV function \verb|cv::connectedComponentsWithStats()|. Only the component with the second maximum area was considered (the first should be the background): it was approximated with a rectangle and everything out of the bounding box was discarded.

		\subsubsection{Dynamic segmentation with depth histograms}
		We also developed an algorithm that was able to dynamically segment depth images with very good precision traded off against very long computation times. \\
		To evaluate this new algorithm we gave a judgment of the segmentation quality on all the 338 images of the first dataset. Judgments were given with boolean values in reference to goodness of the segmentation of the main object represented in the pictures (in this case a person).	\\
		Results gave 92.3\% of precision. Anyway, the longer computational time was not worth of the precision improvement (only 2\% more than the face-detection-first algorithm). However, this could be an improvement if very different images are used. \\
		The algorithm was based on a histogram computation of depth values and on the assumption that it was made of "hills", like a lots of gaussians juxtaposed. It looked for the most frequent value trying to estimate the range containing the highest "hill". If no face was detected in this range it was re-executed on the second most high "hill" and so on. Estimation of the range was made analyzing the logarithm of the second derivative, using fixed thresholds and gradually enlarging the range. Maybe the most frequent value could be replaced with the most large range, or the most wide area in a range.

	\subsection{Face pose estimation and face cropping}	\label{sec:cropping}
	In \citep{Hayat2016}, authors used the algorithm proposed in \citep{Fanelli2011} to compute a face detection and pose estimation. We used the same algorithm in the same way. However we chose to change some of the parameters proposed in \citep{Hayat2016} for face cropping. \\
	
	We instead choosed to [FILL IN] ------------------------------------ ------------------------------------------------------------------- \\

\subsection{Image sets representation}
\label{sec:covariances}
Next steps expected to clusterize pose estimation converted from Euler angles to
rotation matrix and to represent each cluster of each person with the covariance
matrix described below.
\\
We followed all of these steps, but we did not convert Euler angles to rotation
matrix; instead, we clusterized just those Euler angles.
\\
\\
After having divided poses informations through \textit{k-means} in $c$
clusters (with $c=3$), we ended up with $2\times c$ sets of images: $c$ sets of RGB
images and $c$ sets of depth images. Next, the algorithm execute these steps
for each set:
\begin{itemize}
	\item divide each cropped image $j$ in $4\times 4$ non overlapping and equally
		spaced distinct blocks
	\item for each block, compute the LBP representation
	\item compute the difference $y(k, j)$ of each $k$-th LBP vector from
		its own mean
	\item compute a $16\times16$ covariance matrix by summing up all products
		between the standard deviations; in formula:

		$$
		C_{p, q} = \frac{1}{n} \sum_{i=1}^n y(p, i)y(q, i)
		$$

		where $i$ is the index of the image, $p$ and $q$ are
		respectively the row and the column index of the covariance
		matrix.
\end{itemize}
Authors of the proposed paper say that this approach to image set representation
is really effective and that, compared to previous usage of covariance based set
representation \citep{Dai2012}, they overcame some issues dued to a $400\times 400$
very huge matrix covariance.

\subsection{SVM training and prediction}
\label{sec:training}
\cite{Hayat2016} is not very clear in the algorithm used to train
the SVMs. We argued the following:
\begin{itemize}
	\item they created a matrix of SVMs with a row for each different person
		and $2\times c$ colums for each person
	\item every covariance matrix is associated to a SVM
	\item every SVM is trained with only one item labeled $1$ and all other
		items labeled $-1$
	\item every SVM is queried with a covariance $16\times 16$ matrix representing
		a query RGB-D image set
\end{itemize}
Authors merged the predictions of all the $2\times n \times c$ SVMs ($n$ is the number of
identities) using the following strategy:
\begin{itemize}
	\item for each of the $2\times c$ SVMs of the person $i$ that classify the
		query set as being the person $i$, person $i$ earn a vote
	\item the identity choosed is the one with the maximum number of votes
	\item if more than one identity has the maximum number of votes, the
		one with the maximum distance from the hyper-plane is choosed
	\item if no SVM classify the query covariance matrix with positive label,
		it is classified as `unknown'
\end{itemize}

We followed the same strategy. To train the SVMs we used the default grid
provided by OpenCV with logstep. We evaluated each combination of parameters of
each SVM with F-measure, so that only one true positive item existed. We used
the mean of the parameters that performed best. We have also verified that the
mean performed good as well with reference to the F-measure.
\\
The proposed paper did not specify any particular training strategy. Because of
this, we tried three different algorithms:
\begin{itemize}
	\item a simple training in which all covariance matrixes but one had
		negative labels, as described above
	\item a training in which all SVMs of the same correct identity had
		positive label
	\item a training (only on the first dataset) in which the covariance
		matrix of the correct identity but of different poses were
		removed from the trainig set
	\item \textit{leave-one-out}: only on the first datase, without
		success; authors say that \textit{k-fold} cross
		validation gave low results
\end{itemize}
The strategy to merge results from all the $2 \times n \times c$ SVMs was kept
the same too, but we introduced a rule to achieve a better recognition of
"unknown" faces: if more than $t$ SVMs had the maximum number of votes, the
face was forced to be "unknown". $t$ is a threshold that we setted to $2 \times
c$, that is the maximum number of votes that each person should receive. We
checked that this rule was effective: in our experiments it improved the final
algorithm between the 10\% and the 40\% of \textit{rank-1}, while
\textit{FP-unknown} was always 1.
\\
The type of SVMs tested was the only one proposed by \citep{Hayat2016}, that
was based on the following Stein Kernel:
$$
k(X, Y) = e^{\sigma S(X,Y)}
$$
Where
$$
S(X, Y) = \log\det\left(\frac{X + Y}{2}\right)-\frac{1}{2} \log \det(X\times Y)
$$
This is similar to the RBF kernel function and it accepts the same $\gamma$
parameter. Hence, SVMs were trained looking for the two optimal parameters
$\gamma$ and $C$.



\section{Experiments}
With the first small dataset we made many experiments without succeed. \\
From the second dataset we randomly extracted 40 images from each person to be used as testing set and 5 persons were completely removed from the training set, to simulate the `unknown' behaviour. Also, we used $1/3$ of the remaining training set as validation set to evaluate SVMs performance and to find the best hyper-parameters. \\
As training strategy we used only the simple training described at the first point of \ref{sec:training}. Note that maybe a \textit{k-fold cross validation} could enhance the results.

\section{Conclusions and results}
Our results differ a lot based on which identities were removed from the dataset. In no one case we removed the persons that have been recorded twice and for whom there are two sets of images. \\
All this let us argue that our results still suffer from a too much little dataset. Probably, a bigger dataset is needed to perform this algorithm. We also argue that it is not the bigger number of image itself that would made the algorithm perform better on another dataset, but a larger number of different identities, because the avarage number of images per identity in the dataset from \citep{Fanelli2013} (about 700 images per identity) is much larger than in the final dataset used by the authors (about 300 images per identity). \\
Actually, a bigger number of identities means a bigger number of covariance matrixes and a bigger number of items that we can use to train the SVMs. Also, this is coherent with Hayat's statement according to which the covariance set representation is very effective, in that it describes each person in a pose without needing a larger number of images. \\
Also, we note that our results with only RGB or only D images are much lower than those claimed by the authors. We believe that in small datasets, the depth information is more representative than in big datasets. \\
In the following table we show our results achieved on multiple executions of the experiment with different combinations of `unknown'-`known' identities. Results are in terms of:
\begin{itemize}
	\item \textit{rank-1}: ratio between correct recognitions and the total number of sets in the query, where `unknown' identities are considered correct if detected as `unknown';
	\item \textit{FP-unknown}: ratio between uncorrect recognition of `unknown' persons and the total number of `unknown' persons.
\end{itemize}

\begin{table}[]
	\begin{adjustwidth}{-.75in}{-.75in}
		\centering
		\caption{Results}
		\label{tab:results}
		\begin{tabular}{|l|l|c|l|c|l|c|}
			\hline
			\multirow{2}{*}{\bf Removed ids} & \multicolumn{2}{c|}{Using RGB-D images} &\multicolumn{2}{c|}{Using RGB images only} &\multicolumn{2}{c|}{Using D images only} \\ \cline{2-7}
							 & \bf Rank-1 & \bf FP-unknown            & \bf Rank-1 & \bf FP-unknown              & \bf Rank-1 & \bf FP-unknown  \\ \hline
			04, 06, 10, 11, 19               & 0.83	      & 0.4                       & 0.29       & 0.0                         & 0.33       & 0.4\\ \hline
			06, 10, 19, 20, 24               & 0.87	      & 0.0                       & 0.33       & 0.0                         & 0.50       & 0.0\\ \hline
			08, 09, 14, 17, 24               & 0.79	      & 0.4                       & 0.375      & 0.0                         & 0.33       & 0.2\\ \hline
			01, 16, 17, 19, 24               & 0.92	      & 0.2                       & 0.375      & 0.0                         & 0.42       & 0.2\\ \hline
			01, 09, 12, 16, 19               & 0.83	      & 0.6                       & 0.25       & 0.0                         & 0.375      & 0.4\\ \hline
			09, 10, 16, 19, 24               & 0.67	      & 0.8                       & 0.29       & 0.0                         & 0.125      & 0.8\\ \hline
			04, 09, 10, 11, 16               & 0.92	      & 0.2                       & 0.375      & 0.0                         & 0.33       & 0.2\\ \hline
			01, 08, 09, 10, 19               & 0.50	      & 0.8                       & 0.375      & 0.0                         & 0.125      & 0.6\\ \hline
			04, 10, 11, 13, 24               & 0.54	      & 1.0                       & 0.375      & 0.0                         & 0.20       & 0.0\\ \hline
			08, 14, 17, 23, 24               & 0.75	      & 0.6                       & 0.375      & 0.0                         & 0.29       & 0.6\\ \hline
			\bf mean                         & \bf 0.76   & \bf 0.5                   & \bf 0.3785 & \bf 0.0                     & \bf 0.33   & \bf 0.34\\ \hline
		\end{tabular}
	\end{adjustwidth}
\end{table}

\section{Technical issues}
During the developement we found many issues dued to a beta stage of developement in OpenCV. Also, we began using both OpenCV for RGB images and Point Cloud Library for depth images. Since algorithm \citep{Fanelli2011} needed depth images represented in OpenCV \textit{Mat} objects, we needed to switch completely to OpenCV and to leave PCL. \\
Major issues were found in the use of machine learning module of OpenCV, namely in kmeans algorithm and in loading and saving from file of custom kernel SVMs.

\printbibliography

\end{document}
