\documentclass{beamer}
\usetheme{metropolis}
\usepackage{textpos}
\usepackage{multirow}
\usepackage[utf8x]{inputenc}
\definecolor{dei}{RGB}{209, 0, 86}

\title{Face recognition through RGB-D images and matrices of SVMs} % Article title
\subtitle{Based on Hayat et al., 2016}
\date{\today}
\author{Federico Simonetta, matricola 1129912 \\[1ex] Albert Cenzato, matricola 0000000 }
\setbeamercolor{title separator}{fg=dei}
\setbeamercolor{progress bar}{fg=dei}

\begin{document}
\titlegraphic{\vspace{4.3cm}
	\hspace{8.5cm}
	\includegraphics[width=1.5cm]{unipd}

	\hspace{8.5cm}
	\includegraphics[width=2cm]{dei-logo}
}
\maketitle
\addtobeamertemplate{frametitle}{}{\begin{textblock*}{100mm}(.85\textwidth,-1cm)
		\includegraphics[height=1cm,width=2cm]{dei-logo}
	\end{textblock*}
}

\setbeamertemplate{section in toc}{\textbullet\ \normalsize \inserttocsection \\}
\setbeamertemplate{subsection in toc}{\-\hspace{2em}\textbullet\ \small \inserttocsubsection \\ }
\begin{frame}{Overview}
	\vspace*{6mm}
	\tableofcontents[sections, subsections]
\end{frame}

\section{Hayat et al., 2016: \textit{An RGB–D based image set classification
for robust face recognition from Kinect data}}
\begin{frame}{Purpose of the algorithm}
	\begin{itemize}
		\item Purpose of the proposed algorithm is to recognize persons
			given an image set of that person in different poses.
		\item This is useful especially for face recognition in video
			recording.
		\item The algorithm take advantage of depth info captured
			through Kinect cameras.
	\end{itemize}
	\vspace*{-8mm}

\end{frame}
\subsection{Preprocessing}
\begin{frame}{Preprocessing}
	\begin{itemize}
		\item An important step is the image preprocessing
		\item First of all, depth images let a simple \textit{kmeans}
			algorithm being enough to background segmentation
		\item Next, a precise cropping can be made, using precise pose
			informations computed on depth images
		\item Pose info are used to adjust cropping window sizes
	\end{itemize}
\end{frame}

\subsection{Image sets representation}
\begin{frame}{Image sets representation}
	\begin{itemize}
		\item Authors use faces poses to discover 3 different clusters
			for each original image set
		\item They represent each of these clusters with a $16 \times
			16$ covariance matrix computed through LBP features
			vectors
			\begin{itemize}
				\item Each image is subdivided in 16 blocks of
					equal size through a $4 \times 4$ grid
				\item Over each block is computed the LBP
					feature vector
				\item Each entry of the covariance is computed
					on all the LBP of the same block of the
					person in the same pose:
					$$
					C_{p, q} = \frac{1}{n} \sum_{i=1}^n
					y(p, i)y(q, i)
					$$
					where $y(k, j)$ is the difference of
					each $k$-th LBP vector from
					its own mean

			\end{itemize}
	\end{itemize}
\end{frame}
\subsection{SVMs training and prediction}
\begin{frame}{Training}
	\begin{itemize}
		\item Atuhors are not very clear in the training description
		\item They used Stein function as kernel for a vector of SVMs
		\item Each SVM is charged of recognizing one person in one of
			the three poses
		\item For each person in each pose, two SVMs exist, one for RGB
			images and one for depth images
		\item Stein kernel is defined by this equation:
			$$
			k(X, Y) = e^{\sigma S(X,Y)}
			$$
			where
			$$
			S(X, Y) = \log\det\left(\frac{X +
			Y}{2}\right)-\frac{1}{2} \log \det(X\times Y)
			$$
		\item They also tested \textit{k-fold} testing showing that
			results decrease increasing $k$

	\end{itemize}
\end{frame}

\begin{frame}{Prediction}
	\begin{itemize}
		\item A query is composed by a set of images that are
			preprocessed and clusterized in three sets based on
			poses
		\item On each subset the covariance matrix is computed and is
			given in input to each SVM model
		\item Each of the three covariance matrixes of the query will
			receive $6 \times n$ votes, where $n$ is the umber of
			SVMs training sets
		\item The one that will receive the maximum number ov votes
			will be predicted as the recognized face
		\item If no SVM will recognize it, it will be labeld as
			`unknown'
	\end{itemize}
\end{frame}
\section{Datasets}
\section{Our implementation}
\begin{frame}
\end{frame}
\subsection{Preprocessing Boost}
\begin{frame}
\end{frame}
\subsection{Code implementation and design patterns}
\begin{frame}
\end{frame}
\subsection{Implementation issues}
\begin{frame}
\end{frame}
\section{Evaluation}
\subsection{Description of the experiments}
\begin{frame}
\end{frame}
\subsection{Results}
\begin{frame}
\end{frame}
\subsection{Discussion and comparison}
\begin{frame}
\end{frame}
\section{Future Development}
\begin{frame}
\end{frame}

\section*{Thank you for your kind attention}

\end{document}
