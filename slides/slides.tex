\documentclass{beamer}
\usetheme{metropolis}
\usepackage{textpos}
\usepackage{multirow}
\usepackage[utf8x]{inputenc}
\definecolor{dei}{RGB}{209, 0, 86}

\title{Face recognition through RGB-D images and matrices of SVMs} % Article title
\subtitle{Based on Hayat et al., 2016}
\date{\today}
\author{Federico Simonetta, matricola 1129912 \\[1ex] Albert Cenzato, matricola 0000000 }
\setbeamercolor{title separator}{fg=dei}
\setbeamercolor{progress bar}{fg=dei}

\begin{document}
\titlegraphic{\vspace{4.3cm}
	\hspace{8.5cm}
	\includegraphics[width=1.5cm]{unipd}

	\hspace{8.5cm}
	\includegraphics[width=2cm]{dei-logo}
}
\maketitle
\addtobeamertemplate{frametitle}{}{\begin{textblock*}{100mm}(.85\textwidth,-1cm)
		\includegraphics[height=1cm,width=2cm]{dei-logo}
	\end{textblock*}
}

\setbeamertemplate{section in toc}{\textbullet\ \normalsize \inserttocsection \\}
\setbeamertemplate{subsection in toc}{\-\hspace{2em}\textbullet\ \small \inserttocsubsection \\ }
\begin{frame}{Overview}
	\vspace*{3mm}
	\tableofcontents[sections, subsections]
\end{frame}

\section{Hayat et al., 2016: \textit{An RGB–D based image set classification
for robust face recognition from Kinect data}}
\begin{frame}{Purpose of the algorithm}
	\begin{itemize}
		\item Purpose of the proposed algorithm is to recognize persons
			given an image set of that person in different poses.
		\item This is useful especially for face recognition in video
			recording.
		\item The algorithm take advantage of depth info captured
			through Kinect cameras.
	\end{itemize}
	\vspace*{-8mm}

\end{frame}
\subsection{Preprocessing}
\begin{frame}{Preprocessing}
	\begin{itemize}
		\item An important step is the image preprocessing
		\item First of all, depth images let a simple \textit{kmeans}
			algorithm being enough to background segmentation
		\item Next, a precise cropping can be made, using precise pose
			informations computed on depth images
		\item Pose info are used to adjust cropping window sizes
	\end{itemize}
\end{frame}

\subsection{Image sets representation}
\begin{frame}{Image sets representation}
	\begin{itemize}
		\item Authors use faces poses to discover 3 different clusters
			for each original image set
		\item They represent each of these clusters with a $16 \times
			16$ covariance matrix computed through LBP features
			vectors
			\begin{itemize}
				\item Each image is subdivided in 16 blocks of
					equal size through a $4 \times 4$ grid
				\item Over each block is computed the LBP
					feature vector
				\item Each entry of the covariance is computed
					on all the LBP of the same block of the
					person in the same pose:
					$$
					C_{p, q} = \frac{1}{n} \sum_{i=1}^n
					y(p, i)y(q, i)
					$$
					where $y(k, j)$ is the difference of
					each $k$-th LBP vector from
					its own mean

			\end{itemize}
	\end{itemize}
\end{frame}
\subsection{SVMs training and prediction}
\begin{frame}{Training}
	\begin{itemize}
		\item Atuhors are not very clear in the training description
		\item They used Stein function as kernel for a vector of SVMs
		\item Each SVM is charged of recognizing one person in one of
			the three poses
		\item For each person in each pose, two SVMs exist, one for RGB
			images and one for depth images
		\item Stein kernel is defined by this equation:
			$$
			k(X, Y) = e^{\sigma S(X,Y)}
			$$
			where
			$$
			S(X, Y) = \log\det\left(\frac{X +
			Y}{2}\right)-\frac{1}{2} \log \det(X\times Y)
			$$
		\item They also tested \textit{k-fold} testing showing that
			results decrease increasing $k$

	\end{itemize}
\end{frame}

\begin{frame}{Prediction}
	\begin{itemize}
		\item A query is composed by a set of images that are
			preprocessed and clusterized in three sets based on
			poses
		\item On each subset the covariance matrix is computed and is
			given in input to each SVM model
		\item Each of the three covariance matrixes of the query will
			receive $6 \times n$ votes, where $n$ is the umber of
			SVMs training sets
		\item The one that will receive the maximum number ov votes
			will be predicted as the recognized face
		\item If no SVM will recognize it, it will be labeld as
			`unknown'
	\end{itemize}
\end{frame}

\section{Datasets}
\begin{frame}{Datasets}
	\begin{itemize}
		\item The project assignment was to test the proposed algorithm
			on a new small dataset of 338 images coming from 26
			people, with different lightings and background scenes
		\item This was unsuccesful, hence, after many attempts, we
			changed the dataset and we tested the algorithm on a
			subset of the dataset used by the authors
		\item This new dataset contains more than 15.000 RGB-D images
			with 24 different images sets of 20 different peoples
	\end{itemize}
\end{frame}

\section{Our implementation}
\subsection{Proprocessing tweaks}
\begin{frame}{Preprocessing tweaks (1)}
	The very poor results on the first small dataset
	forwarded us to implement some changes to the
	preprocessing algorithm
	\begin{itemize}
		\item \textit{face-detection-first}: we used Haar cascade
			classifiers made available by OpenCV to detect a face,
			crop it and then segment background and refine
			cropping; if no face was detected, we use a fixed
			threshold segmentation.
		\item \textit{outlier removal}: we first compute the maximum
			areconnected component and crop it with its smallest
			rectangle approximation; this is useful to a better
			clustering
	\end{itemize}
\end{frame}

\begin{frame}{Preprocessing tweaks (2)}
	\begin{itemize}
		\item \textit{histograms segmentation}: we also develped a new
			algorithm based on the assumption that depth histograms
			envelope is similar to several gaussians or `hills'
			justaxposed; we tried to segment the most wide `hill'
			and if face detection or pose estimation failed we went
			through the second wider `hill'. Despite this algorithm
			was very effective, we decided to abandon it because of
			its huge computational cost.
		\item \textit{adjusted face cropping}:
			------------FILL-------------------FILL-------
		\item \textit{pose clustering}:
			------------FILL-------------------FILL-------
	\end{itemize}
\end{frame}

\subsection{Training strategies}
\begin{frame}
	The proposed paper did not specify any particular training strategy.
	Because of this, we tried three different algorithms:
	\begin{itemize}
		\item a simple training in which all covariance matrixes but
			one had negative labels
		\item a training in which all SVMs of the same correct identity
			had positive label (very bad results)
		\item a training (only on the first dataset) in which the
			covariance matrix of the correct identity but of
			different poses were removed from the trainig set
		\item \textit{leave-one-out}: only on the first datase, without
			success; authors say that \textit{k-fold} cross
			validation gave low results
	\end{itemize}
\end{frame}

\section{Technical issues}
\subsection{Computational issues}
\begin{frame}{Computational issues}
	\begin{itemize}
		\item Preprocessing is really slow, especially the covariance
			computation
		\item We implemented multithread version but the main advantage
			given by the multithread is in file loading, background
			removal and face cropping because in covariance
			computing --------FILL IN--------
		\item Using Intel i7-4510HQ CPU, (4 cores, 2 threads per core,
			2.5 GHz frequency, boosted to 3.5 GHz), preprocessing
			and covariance computing of all the dataset take about
			10 minutes
		\item There are not particular issue on RAM usage, it is nedded
			about 1GB during preprocessing.
	\end{itemize}
\end{frame}

\subsection{Implementation issues}
\begin{frame}{Implementation issues}
	\begin{itemize}
		\item Code for pose estimation needed depth images represented
			in OpenCV \textit{Mat} objects
		\item We needed to switch completely to OpenCV
		\item Other issues were found in the use of machine learning
			module of OpenCV, namely
			\begin{itemize}
				\item in kmeans algorithm
				\item in loading and saving from file of custom
					kernel SVMs
			\end{itemize}
	\end{itemize}
\end{frame}

\section{Evaluation}
\subsection{Description of the experiments}
\begin{frame}{Description of the experiments}
	We setted many experiments, with different combination of
	`known'-`unknown' ids. Every one was setted up in this way:
	\begin{itemize}
		\item 40 images per person were randomly removed and used as
			testing set. Each query was made by a set of 40 images
		\item 5 peoples choiced randomly were completely removed from
			the training set to simulate the `unkown' behaviour
		\item People who had two recordings was never removed
		\item $1/3$ of each sequence of the training set was used as
			validation set to evaluate SVMs during parameters
			otpimization
	\end{itemize}
\end{frame}

\subsection{Discussion and results}
\begin{frame}{Discussion}
	\begin{itemize}
		\item Our results are comparable to the ones of the authors,
			taking in account the minor size of our dataset
		\item Despite the smaller sizes, the second dataset that we
			used had a very larger mean number of images per
			persons (about 700 images) with respect to the dataset
			used by the authors (about 300 images per identity)
		\item This let us argue that the major contribute to good
			performances is not given by the total number of images,
			but it is given by the number of identities
		\item Our results with only RGB or D images are much lower than
			those claimed by authors. We bielieve that in small
			datasets depth info is more representative than in
			bigger datset.
	\end{itemize}
\end{frame}

\begin{frame}{Results (1)}
	We computed results in terms of:
	\begin{itemize}
		\item \textit{rank-1}: ratio between correct recognitions and
			the total number of sets in the query, where `unknown'
			identities are considered correct if detected as
			`unknown'
		\item \textit{FP-unknown}: ratio between uncorrect recognition
			of `unknown' persons and the total number of `unknown'
			persons
	\end{itemize}
\end{frame}

\begin{frame}{Results (2)}
	\begin{table}[]
		\centering
		\caption{Results with RGB and depth images}
		\label{tab:results-rgbd}
		\begin{tabular}{|l|c|}
			\hline
			\bf Rank-1 & \bf FP-unknown \\ \hline
			0.89	   & 0.8 \\ \hline
			0.83	   & 0.8 \\ \hline
			0.75	   & 0.3 \\ \hline
			0.83	   & 0.6 \\ \hline
			0.875      & 0.8 \\ \hline
		\end{tabular}
	\end{table}

\end{frame}

\section{Future Development}
\begin{frame}
	????
\end{frame}

\section*{Thank you for your kind attention}

\end{document}
